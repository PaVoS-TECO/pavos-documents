\chapter{Funktionale Anforderungen}
\begin{enumerate}[label=\textbf{PF\arabic{enumi}0}]
	\setcounter{enumi}{99}
	\section{Pflichtanforderungen}
	\subsection{Server}

	\item \textbf{Initialisierendes Starten}\\
		Beim ersten Aufruf wird der Nutzer aufgefordert, einen Kafka-Stream in die Konfigurations-GUI einzufügen. Die Gültigkeit dieses Streams wird überprüft. [\ref{Gültig}]\\
		\textbf{Vorbedingung:} Programm wird gestartet.\\
		\textbf{Nachbedingung:} Das Programm ist einsatzbereit.\\
		\textbf{Fehlerbedingung:} Das Programm kann nicht weiter genutzt werden. Der Nutzer wird erneut aufgefordert, einen Kafka-Stream einzufügen.
	
	\item \textbf{Kontrollieren der Gültigkeit des Streams} \label{Gültig}\\
		Es wird eine Verbindung zu Kafka hergestellt und die syntaktische Korrektheit des Streams wird überprüft.\\
		\textbf{Vorbedingung:} Ein Kafka-Stream wurde hinzugefügt. [\ref{Einfügen}]\\
		\textbf{Nachbedingung:} Der Kafka-Stream wird vom Server akzeptiert.\\
		\textbf{Fehlerbedingung:} Der Kafka-Stream wird nicht vom Server akzeptiert. Der Nutzer wird erneut aufgefordert, einen Kafka-Stream einzufügen.
	
	\item \textbf{Einfügen eines Kafka-Streams} \label{Einfügen}\\
		Ein Kafka-Stream kann in der Konfigurations-GUI eingefügt werden. Die Gültigkeit dieses Streams wird überprüft. [\ref{Gültig}]\\
		\textbf{Vorbedingung:} Das Programm ist gestartet.\\
		\textbf{Nachbedingung:} Der Kafka-Stream wird dem Programm hinzugefügt.\\
		\textbf{Fehlerbedingung:} Der Kafka-Stream wird nicht hinzugefügt. Der Nutzer kann erneut einen Kafka-Stream einfügen.
		
	\item \textbf{Entfernen eines Kafka-Streams}\\
		Ein Kafka-Stream kann in der Konfigurations-GUI entfernt werden.\\
		\textbf{Vorbedingung:} Es ist mindestens ein Kafka-Stream vorhanden.\\
		\textbf{Nachbedingung:} Der Kafka-Stream wird aus dem Programm entfernt. Falls der letzte Stream entfernt wird, stoppt das Programm.\\
		\textbf{Fehlerbedingung:} Der Kafka-Stream wird nicht entfernt. Dem Nutzer wird darauf hingewiesen.
		
		% Zu trivial
		
%	\item \textbf{Ein und Ausschalten des Programms}\\
%		Über die Konfigurations-GUI kann das Programm gestoppt und gestartet werden.\\
%		\textbf{Vorbedingung:} Das Programm läuft\\
%		\textbf{Nachbedingung:} Das Programm wird gestoppt oder gestartet.\\
%		\textbf{Fehlerbedingung:} Nichts passiert und es wird eine Fehlermeldung ausgegeben.
		
	\item \textbf{Importieren von NetCDF-Dateien}\\
		Eine \gls{netcdf_glo}-Datei kann importiert werden, woraus ein Kafka-Stream erstellt wird. Dieser wird automatisch zum Programm hinzugefügt. [\ref{Einfügen}]\\
		\textbf{Vorbedingung:} Das Programm ist gestartet.\\
		\textbf{Nachbedingung:} Die Sensordaten werden verarbeitet.\\
		\textbf{Fehlerbedingung:} Die Daten werden nicht importiert. Dem Nutzer wird eine Fehlermeldung angezeigt.
		
	\item \textbf{Importieren von CSV-Dateien}\\
		Eine \gls{csv}-Datei kann importiert werden, woraus ein Kafka-Stream erstellt wird. Dieser wird automatisch zum Programm hinzugefügt. [\ref{Einfügen}]\\
		\textbf{Vorbedingung:} Das Programm ist gestartet.\\
		\textbf{Nachbedingung:} Die Sensordaten werden verarbeitet.\\
		\textbf{Fehlerbedingung:} Die Daten werden nicht importiert. Dem Nutzer wird eine Fehlermeldung angezeigt.
		
	\item \textbf{Sensoren in Tabellenansicht}\\
		Die einzelnen Sensoren eines Streams können in der Konfigurations-GUI in einer Tabelle anzeigt werden.\\
		\textbf{Vorbedingung:} Es ist mindestens ein Kafka-Stream vorhanden.\\
		\textbf{Nachbedingung:} Die Sensoren werden in einer Tabelle angezeigt.\\
		\textbf{Fehlerbedingung:} Dem Nutzer wird eine Fehlermeldung angezeigt.
		
	\item \textbf{Sensoren nach Parametern sortieren}\\
		Der Nutzer kann die Sensoren in der Tabellenansicht nach ihren Parametern sortieren.\\
		\textbf{Vorbedingung:} Es werden Sensoren in der Tabelle aufgelistet.\\
		\textbf{Nachbedingung:} Die Sensoren werden nach dem ausgewählten Parameter sortiert.\\
		\textbf{Fehlerbedingung:} Dem Nutzer wird eine Fehlermeldung angezeigt.
		
	\item \textbf{Aufteilung der Daten} \label{Aufteilung}\\
		Die Sensordaten werden in skalar- und vektorwertige Daten aufgeteilt.\\
		\textbf{Vorbedingung:} Es ist mindestens ein Kafka-Stream vorhanden.\\
		\textbf{Nachbedingung:} Die Sensordaten werden weiterverarbeitet.\\
		\textbf{Fehlerbedingung:} Die Daten werden nicht aufgeteilt. Im Fehlerprotokoll wird ein Fehler vermerkt.
		
	\item \textbf{Berechnung des Mittelwerts} \label{Mittel}\\
		Von den aufgeteilten Sensordaten werden Mittelwerte berechnet. [\ref{Aufteilung}]\\
		\textbf{Vorbedingung:} Sensordaten sind aufgeteilt.\\
		\textbf{Nachbedingung:} Die Mittelwerte können an den Webdienst weitergeleitet werden.\\
		\textbf{Fehlerbedingung:} Es werden keine Mittelwerte berechnet. Im Fehlerprotokoll wird ein Fehler vermerkt.
		
	\item \textbf{Kafka-Streams werden aufgeteilt}\\
		Die Kafka-Streams werden auf verschiedene Worker-Dienste aufgeteilt um dort weiterverarbeitet zu werden.\\
		\textbf{Vorbedingung:} Es ist mindestens ein Kafka-Stream vorhanden.\\
		\textbf{Nachbedingung:} Die Daten sind erfolgreich aufgeteilt und werden verarbeitet.\\
		\textbf{Fehlerbedingung:} Im Fehlerprotokoll wird ein Fehler vermerkt.
		
	\item \textbf{Clusterform auswählen}\\
		Die Form der Cluster kann in der Konfigurations-GUI eingestellt werden. Die Standardform ist ein Rechteck.\\
		\textbf{Vorbedingung:} Es ist mindestens ein Kafka-Stream vorhanden.\\
		\textbf{Nachbedingung:} Die Clusterform wird angepasst.\\
		\textbf{Fehlerbedingung:} Dem Nutzer wird eine Fehlermeldung angezeigt.
		
	\item \textbf{Cluster berechnen} \label{Kachel}\\
		Anhand der Form des Clusters werden die Cluster berechnet und an den Webdienst weitergeleitet.\\
		\textbf{Vorbedingung:} Es ist mindestens ein Kafka-Stream vorhanden.\\
		\textbf{Nachbedingung:} Die Cluster werden an den Webdienst weitergeleitet.\\
		\textbf{Fehlerbedingung:} Dem Nutzer wird eine Fehlermeldung angezeigt.
	
	\item \textbf{Cluster werden an den Webdienst gesendet} \label{send}\\
		Die Cluster werden an den Webdienst gesendet um im Webinterface dargestellt zu werden.\\
		\textbf{Vorbedingung:} Clusterposition, Clustergröße und Clusterform wurden berechnet.\\
		\textbf{Nachbedingung:} Die Cluster werden vom Webdienst entgegengenommen.\\
		\textbf{Fehlerbedingung:} Im Fehlerprotokoll wird ein Fehler vermerkt.
	
	\item \textbf{Export der Daten als NetCDF-Datei} \label{netcdf}\\
		Die Sensordaten aus einem bestimmten zeitlichen Intervall können im NetCDF-Format exportiert werden.\\
		\textbf{Vorbedingung:} Es ist mindestens ein Kafka-Stream vorhanden.\\
		\textbf{Nachbedingung:} Die Daten werden exportiert.\\
		\textbf{Fehlerbedingung:} Dem Nutzer wird eine Fehlermeldung angezeigt.
	
	\item \textbf{Export der Daten als CSV-Datei} \label{CSV}\\
		Die Sensordaten aus einem bestimmten zeitlichen Intervall können im CSV-Format exportiert werden.\\
		\textbf{Vorbedingung:} Es ist mindestens ein Kafka-Stream vorhanden.\\
		\textbf{Nachbedingung:} Die Daten werden exportiert.\\
		\textbf{Fehlerbedingung:} Dem Nutzer wird eine Fehlermeldung angezeigt.
		
	%%Client Worker
	
%%%%%	\item \textbf{Daten werden an Webinterface gesendet} \label{FAWebsend}\\
	%	Die Daten aus dem Datenerhaltungssystem und vom Kafka-Stream können an das %Webinterface gesendet werden.
	 	
	% \item \textbf{Hinzufügen weiter Module}\\
	%	 Der Dienst ist so aufgebraucht, dass es dem Nutzer und Programmierer möglich sein %wird, auf dem einzelnen Interface zuzugreifen und so neue Module für den Dienst zu entwickeln um das System zu erweitern. Ein Beispiel wäre neue Exportformate hinzuzufügen oder ein Modul zwischen Server-Dienst und Webinterfaces zu setzten. 
	 
%	 \item \textbf{Der Dienst unterstützt skalar- und vektorwertige Sensortypen}
	% 	Der Dienst unterstützt beim Kafka-Stream und beim Import von Sensor Daten nur Skalar Sensortypen und vekorwertige Sensortypen. Falls andere Sensortypen im Stream entdeckt werden, dann wird dem Nutzer eine Anleitung angezeigt wie er selbst ein neuen Sensortyp hinzufügen kann.
 
	 \subsection{Webinterface}
	 \setcounter{enumi}{199}
	 
	 \item \textbf{Mehrere Instanzen des Webinterface}\\
		 Mehrere Instanzen des Webinterfaces können aufgerufen werden, sodass mehrere Nutzer gleichzeitig mit den Sensordaten arbeiten können.\\
		 \textbf{Vorbedingung:} Es ist mindestens ein Kafka-Stream vorhanden.\\
		 \textbf{Nachbedingung:} Mehrere Instanzen des Webinterfaces sind geöffnet.\\
	 	\textbf{Fehlerbedingung:} Im Fehlerprotokoll wird ein Fehler vermerkt. Dem Nutzer des Webinterfaces wird eine Fehlermeldung angezeigt.

	 \item \textbf{Sensordaten werden auf der Karte dargestellt}\\
	 	Nachdem die Daten an den Webdienst gesendet wurden [\ref{send}], werden sie auf der Karte des Webinterfaces angezeigt.\\
	 	\textbf{Vorbedingung:} Daten wurden an den Webdienst gesendet.\\
	 	\textbf{Nachbedingung:} Die Daten werden im Webinterface angezeigt.\\
	 	\textbf{Fehlerbedingung:} Im Fehlerprotokoll wird ein Fehler vermerkt. Dem Nutzer des Webinterfaces wird eine Fehlermeldung angezeigt.
	 	
	\item \textbf{Echtzeitdaten}\\
		Der Nutzer kann sich die Echtzeitdaten für einen Cluster oder Sensor anzeigen lassen\\
		\textbf{Vorbedingung:} Es ist mindestens ein Kafka-Stream vorhanden.\\
		\textbf{Nachbedingung:} Die Echtzeitdaten werden angezeigt.\\
		\textbf{Fehlerbedingung:} Im Fehlerprotokoll wird ein Fehler vermerkt.
	 
	 \item \textbf{Verschiedene Sensortypen darstellen}\\
	 	Die Darstellung wird in unterschiedliche Sensortypen aufgeteilt.\\
	 	\textbf{Vorbedingung:} Es werden Sensordaten angezeigt.\\
	 	\textbf{Nachbedingung:} Es werden Sensordaten zum Sensortyp angezeigt.\\
	 	\textbf{Fehlerbedingung:} Im Fehlerprotokoll wird ein Fehler vermerkt.
	 
	 \item \textbf{Verschiedene Clusterformen}\\
	 	Der Nutzer kann die Karte in unterschiedliche Clusterformen einteilen.\\
	 	\textbf{Vorbedingung:} Es werden Sensordaten angezeigt.\\
	 	\textbf{Nachbedingung:} Die Einteilung in eine Clusterform wird angezeigt.\\
	 	\textbf{Fehlerbedingung:} Im Fehlerprotokoll wird ein Fehler vermerkt.
		
	\item \textbf{Export von Daten}\\
		Der Nutzer kann die Daten im CSV oder NetCDF-Format exportieren.[\ref{netcdf}] [\ref{CSV}]\\
		\textbf{Vorbedingung:} Es ist mindestens ein Kafka-Stream vorhanden.\\
		\textbf{Nachbedingung:} Die Daten werden exportiert.\\
		\textbf{Fehlerbedingung:} Dem Nutzer wird eine Fehlermeldung angezeigt.
		
	\item \textbf{Wiederholung anzeigen} \label{wiederhol}\\
		Auf dem Webinterface kann man Daten mithilfe eines Buttons als Wiederholung anzeigen lassen.\\
		\textbf{Vorbedingung:} Es werden Sensordaten angezeigt.\\
		\textbf{Nachbedingung:} Die Wiederholung wird angezeigt.\\
		\textbf{Fehlerbedingung:} Im Fehlerprotokoll wird ein Fehler vermerkt. Dem Nutzer des Webinterfaces wird eine Fehlermeldung angezeigt.
		

	\item \textbf{Detailansicht von Sensoren}\\
		Auf dem Webinterface kann man von einzelnen Sensoren eine Detailansicht ansehen um so nähere Informationen über den Sensor zu erhalten. [\ref{single}]\\
		\textbf{Vorbedingung:} Es ist mindestens ein Kafka-Stream vorhanden.\\
		\textbf{Nachbedingung:} Die Detailansicht wird angezeigt.\\
		\textbf{Fehlerbedingung:} Im Fehlerprotokoll wird ein Fehler vermerkt. Dem Nutzer des Webinterfaces wird eine Fehlermeldung angezeigt.
		
	\item \textbf{Statistiken anzeigen}\\
		Auf dem Webinterface kann man sich vorberechnete Statistiken anzeigen lassen [\ref{single}] [\ref{stat}]\\
		\textbf{Vorbedingung:} Es ist mindestens ein Kafka-Stream vorhanden.\\
		\textbf{Nachbedingung:} Die Statistiken werden angezeigt.\\
		\textbf{Fehlerbedingung:} Im Fehlerprotokoll wird ein Fehler vermerkt.
				
	\item \textbf{Zoomen in die Karte}\\
		Auf der Karte kann gezoomt werden. Die Cluster werden entsprechend angepasst.\\
		\textbf{Vorbedingung:} Die Karte wird angezeigt.\\
		\textbf{Nachbedingung:} Die gezoomte Ansicht wird angezeigt.\\
		\textbf{Fehlerbedingung:} Im Fehlerprotokoll wird ein Fehler vermerkt.
		
\end{enumerate}

\begin{enumerate}[label=\textbf{WF\arabic{enumi}0}]
	\setcounter{enumi}{99}
	\section{Wunschanforderungen}
	
	\subsection{Server}
	
	\item \textbf{Suche nach fehlerhaften Sensordaten}\\
	Der Dienst überprüft im Datenerhaltungssystem ob Sensordaten Outlier sein können.\\
	\textbf{Vorbedingung:} Es ist mindestens ein Kafka-Stream vorhanden\\
	\textbf{Nachbedingung:} Im Fehlerprotokoll wird ein Hinweis vermerkt. Fehlerhafte Daten werden angezeigt.\\
		
	\subsection{Webinterface}
	\setcounter{enumi}{199}
	
	%Überarbeiten der Favoriten
	\item \textbf{Favoriten}\\
		Der Nutzer kann seine Ansicht favorisieren um später schnell auf die Ansicht zu wechseln.\\
		\textbf{Vorbedingung:} Die Karte wird angezeigt.\\
		\textbf{Nachbedingung:} Der Favorit wird gespeichert.\\
		\textbf{Fehlerbedingung:} Der Favorit wird nicht gespeichert. Dem Nutzer wird eine Fehlermeldung angezeigt.
	
	\item \textbf{Sensoren melden}\\
		Der Nutzer kann einzelne Sensoren auf dem Webinterface melden. Dazu muss ein Grund angeben werden.\\
		\textbf{Vorbedingung:} Der Nutzer hat einen Sensor selektiert.\\
		\textbf{Nachbedingung:} Im Fehlerprotokoll wird ein Hinweis vermerkt.\\
		
	\item \textbf{Leere Cluster}\\
		Falls ein Cluster keine Daten enthält wird ein leeres Cluster angezeigt.\\
		\textbf{Vorbedingung:} Die Karte wird angezeigt.\\
		\textbf{Nachbedingung:} Es wird ein leeres Cluster angezeigt.\\
		
	\item \textbf{Wiederholung pausieren}\\
		Der Nutzer kann die Wiederholung pausieren und wieder fortsetzen. [\ref{wiederhol}]\\
		\textbf{Vorbedingung:} Die Wiederholung wird angezeigt.\\
		\textbf{Nachbedingung:} Die Wiederholung wird pausiert/fortgesetzt.\\
	
	\item \textbf{Eigene Filter}\\
		Der Nutzer kann eigene Filter erstellen.\\
		\textbf{Vorbedingung:} Es ist mindestens ein Kafka-Stream vorhanden.\\
		\textbf{Nachbedingung:} Filteransicht.\\
		
	\item \textbf{Mehrere Sprachen}\\
		Die Anzeigesprache kann in mehrere Sprachen geändert werden.\\
		\textbf{Vorbedingung:} Das Programm ist gestartet.\\
		\textbf{Nachbedingung:} Sprache wird geändert.\\

\end{enumerate}
