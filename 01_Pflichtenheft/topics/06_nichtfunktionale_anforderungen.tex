\chapter{Nichtfunktionale Anforderungen}
\begin{enumerate}[label=\textbf{NF\arabic{enumi}0}]
	\setcounter{enumi}{99}
	
	\item \textbf{Anzeigesprache}\\
	Die Standardanzeigesprache des Webinterfaces ist Englisch.
	
	\item \textbf{Parallele Zugriffe}\\
	Das Webinterface muss von bis zu 100 Nutzern gleichzeitig bedient werden können. Alle folgenden Anforderungen gelten nur für bis zu 100 parallele Nutzer. Für mehr Nutzer sind keine Leistungsgarantien gegeben.
	
	\item \textbf{Sichtbare Cluster}\\
	Das Webinterface muss bis zu 1000 Cluster gleichzeitig anzeigen können. Alle folgenden Anforderungen gelten nur für bis zu 1000 sichtbare Cluster. Für mehr Cluster sind keine Leistungsgarantien gegeben.
	
	\item \textbf{Verzögerung von neuen Daten}\\
	Neue Echtzeitsensordaten werden innerhalb einer Minute verarbeitet und auf allen Instanzen der Webinterface angezeigt.
	
	\item \textbf{Verzögerung von historischen und archivierten Daten}\\
	Historische und archivierte Daten, die eine große Datenmenge umfassen und/oder von einem persistenten Speicher geladen werden müssen, benötigen mehr Zeit zur Verarbeitung und Visualisierung. Für diese sind keine Leistungsgarantien gegeben.
	
	\item \textbf{Speicherung von kürzlichen Echtzeitsensordaten}\\
	Kürzliche Echtzeitsensordaten werden bis zu einer Gesamtgröße von 1GB in dem Arbeitsspeicher gespeichert. Diese kürzlichen Daten können schneller abgerufen werden als ältere Daten und werden für die Wiederholungsansicht verwendet.
	
	\item \textbf{Speicherung von älteren Echtzeitsensordaten}\\
	Alle an Apache Kafka angelegte Sensordaten werden von Apache Kafka in einem persistenten Speicher gespeichert. Die Größe des Speichers begrenzt die Menge der gespeicherten Daten.
	
	\item \textbf{Skalierbarkeit}\\
	Der Server startet automatisch neue Worker-Container um Sensordaten und Anfragen des Servers zu verarbeiten. Mindestens 10 Worker können je nach Bedarf gestartet und beendet werden.
	
	\item \textbf{Stabilität}\\  
	Der Server muss im normalen Betrieb problemlos mit unerwarteter Terminierung von einem oder mehreren Worker-Containern umgehen und innerhalb von fünf Minuten die ursprüngliche Rechenleistung wiederherstellen. Fehler in den Eingabedaten wie ungültige Werte oder Syntaxfehler führen nicht zu einem Absturz des Servers.
	
	\item \textbf{Reaktionszeit}\\
	Das Webinterface muss innerhalb von zwei Sekunden auf Interaktionen des Nutzers reagieren. Falls der Webserver bis dahin nicht das erwünschte Ergebnis liefern kann, wird der Nutzer darauf hingewiesen.
	
	\item \textbf{Ladezeiten der Darstellungselemente}\\
	Ein Zugriff auf das Webinterface lädt Darstellungselemente wie Cluster-Shapes, Graphen und Buttons innerhalb von 5 Sekunden.
	
	\item \textbf{Genauigkeit der Clusterberechnung}\\
	Die Approximation der Daten in Clustern darf nicht zu einer Abweichung über 5\% von Ergebnissen einer akkuraten Berechnung führen.
	
	\item \textbf{Geschwindigkeit der Clusterberechnung}\\
	Die Approximation der Daten in Clustern darf nicht mehr als 2 Sekunden dauern.
	
	\item \textbf{Aktualisierungsgeschwindigkeit}\\
	Die Karte aktualisiert Echtzeitsensordaten alle 10 Sekunden.

\end{enumerate}
	