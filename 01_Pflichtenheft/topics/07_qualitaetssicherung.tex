\chapter{Qualitätssicherung}
	\label{ch:Qualitaetssicherung}
	
	Für die Komponententests unseres Systems verwenden wir grundsätzlich JUnit. Für weiterführenden Tests werden im folgenden Abschnitt Szenarien beschrieben.
	
\section{Globale Testszenarien}
	\label{ch:Qualitaetssicherung:sec:GlobaleTestszenarien}
	
\subsection{Server}
\begin{enumerate}[label=\textbf{TI\arabic{enumi}0}]
	\setcounter{enumi}{99}
	
	\item \textbf{Erstanwendung der Konfigurations-GUI mit Kafka Stream}\\
		Die Konfigurations-GUI wurde noch nie gestartet und es steht ein \glsdisp{apache_kafka}{Kafka} \glsdisp{datenstrom}{Stream} namens 'TestStream' bereit.\\
		Der Nutzer startet die Konfigurations-GUI, woraufhin diese ihn auffordert, den hinzuzufügenden Kafka Stream auszuwählen.\par
		
		Der Nutzer schreibt 'TestStream' in das Feld des Popups und bestätigt.\\
		Die Konfigurations-GUI akzeptiert den Stream und fügt ihn hinzu.
	\item \textbf{Erstanwendung der Konfigurations-GUI ohne Kafka Stream}\\
		Die Konfigurations-GUI wurde noch nie gestartet und es steht kein Kafka Stream bereit.\\
		Der Nutzer startet die Konfigurations-GUI, woraufhin diese ihn auffordert, den hinzuzufügenden Kafka Stream auszuwählen.\par
		
		Der Nutzer schreibt 'TestStream' in das Feld des Popups und bestätigt.\\
		Die Konfigurations-GUI akzeptiert den Stream nicht und fordert ihn erneut auf, den hinzuzufügenden Kafka Stream auszuwählen.
	\item \textbf{Hinzufügen von zusätzlichen Streams zur Konfigurations-GUI mit Kafka Stream}\\
		Die Konfigurations-GUI ist gestartet, es ist ein Kafka Stream namens 'TestStream' hinzugefügt worden und es steht ein Kafka Stream namens 'TestStream2' bereit.\\
		Der Nutzer wählt die Option 'Stream hinzufügen', woraufhin die Konfigurations-GUI ihn auffordert, den hinzuzufügenden Kafka Stream auszuwählen.\par
		
		Der Nutzer schreibt 'TestStream2' in das Feld des Popups und bestätigt.\\
		Die Konfigurations-GUI akzeptiert den Stream und fügt ihn hinzu.
	\item \textbf{Hinzufügen von zusätzlichen Streams zur Konfigurations-GUI ohne Kafka Stream}\\
		Die Konfigurations-GUI ist gestartet, es ist ein Kafka Stream namens 'TestStream' hinzugefügt worden und es steht kein Kafka Stream bereit.\\
		Der Nutzer wählt die Option 'Stream hinzufügen', woraufhin die Konfigurations-GUI ihn auffordert, den hinzuzufügenden Kafka Stream auszuwählen.\par
		
		Der Nutzer schreibt 'TestStream' in das Feld des Popups und bestätigt.\\
		Die Konfigurations-GUI akzeptiert den Stream nicht und fordert ihn erneut auf, den hinzuzufügenden Kafka Stream auszuwählen.
	\item \textbf{Entfernen von Streams in der Konfigurations-GUI}\\
		Die Konfigurations-GUI ist gestartet und es ist ein Kafka Stream namens 'TestStream' hinzugefügt worden.\\
		Der Nutzer wählt die Option 'Stream entfernen', woraufhin die Konfigurations-GUI ihn auffordert, den zu entfernenden Kafka Stream auszuwählen.\par
		
		Der Nutzer schreibt 'TestStream' in das Feld des Popups und bestätigt.\\
		Die Konfigurations-GUI entfernt den Stream.
	\item \textbf{Export einer Archiv-Datei in der Konfigurations-GUI}\\
		Die Konfigurations-GUI ist gestartet und es ist ein Kafka Stream namens 'TestStream' hinzugefügt worden.\\
		Der Nutzer wählt die Option 'Export Stream', woraufhin die Konfigurations-GUI ihn auffordert, den Dateityp, das Verzeichnis und den Namen der zu exportierenden Archiv-Datei auszuwählen.\par
		
		Der Nutzer wählt das Format
		\begin{itemize}
		\item \gls{netcdf_glo}
		\item \gls{csv}
		\end{itemize}
		, wählt das Verzeichnis 'SensorExport' bzw. erstellt es, falls es noch nicht existiert, legt den Namen auf 'TestSensorExport' fest und bestätigt.\\
		Die Konfigurations-GUI exportiert den Stream.
	\item \textbf{Import einer Archiv-Datei in der Konfigurations-GUI}\\
		Die Konfigurations-GUI ist gestartet, es ist ein Kafka Stream namens 'TestStream' hinzugefügt worden und es existiert eine importierbare Archiv-Datei im Verzeichnis 'SensorExport' mit dem Namen 'TestSensorExport'.\\
		Der Nutzer wählt die Option 'Import Stream', woraufhin die Konfigurations-GUI ihn auffordert, die zu importierenden Archiv-Datei auszuwählen.\par
		
		Der Nutzer wählt die
		\begin{itemize}
		\item NetCDF
		\item CSV
		\end{itemize}
		Datei und bestätigt.\\
		Die Konfigurations-GUI importiert den Stream.
	\item \textbf{\glsdisp{cluster_shape}{Clusterform} in der Konfigurations-GUI festlegen}\\
		Die Konfigurations-GUI ist gestartet und es ist ein Kafka Stream namens 'TestStream' hinzugefügt worden.\\
		Der Nutzer wählt die Option 'Select Clusterform', woraufhin die Konfigurations-GUI ihn auffordert, aus mehreren Möglichkeiten eine Form für die \gls{cluster} zu berechnen.\par
		
		Der Nutzer wählt die
		\begin{itemize}
		\item Quadrat
		\item Hexagon
		\end{itemize}
		Form und bestätigt.\\
		Die Konfigurations-GUI akzeptiert die neue Clusterform und berechnet Mittelwerte und Clusterpositionen neu.

\end{enumerate}

\subsection{Webinterface}
\begin{enumerate}[label=\textbf{TI\arabic{enumi}0}]
	\setcounter{enumi}{199}
	
	\item \textbf{Darstellung des berechneten Mittelwertes im Webinterface}\\
		Das \gls{webinterface} ist gestartet und es ist ein Kafka Stream namens 'TestStream' hinzugefügt worden. Die Ansicht des Webinterfaces ist auf Karlsruhe in Deutschland fokussiert.\\
		Der Nutzer wählt alle Skalierungsstufen der Karte in sich entfernender Reihenfolge aus, woraufhin die berechneten Mittelwerte auf der Karte als Farbe der Cluster zu sehen sind.\par
	\item \textbf{Darstellung von Clustern ohne Sensoren im Webinterface}\\
		Das Webinterface ist gestartet und es ist ein Kafka Stream namens 'TestStream' hinzugefügt worden. Die Ansicht des Webinterfaces ist auf Karlsruhe in Deutschland fokussiert. Es existiert ein Cluster ohne \glsdisp{sensor}{Sensoren} in seinem Einflussgebiet, hier auch 'leerer Cluster' genannt.\\
		Der Nutzer wählt alle Skalierungsstufen der Karte in sich entfernender Reihenfolge aus. Die Karte zeigt leere Cluster nicht an.\par
	\item \textbf{Darstellung der tabellarischen Ansicht im Webinterface}\\
		Das Webinterface ist gestartet, es ist ein Kafka Stream namens 'TestStream' hinzugefügt worden. Die Ansicht des Webinterfaces ist auf Karlsruhe in Deutschland fokussiert.\\
		Der Nutzer wählt in der Ansicht den tabellarischen Modus.\\
		Daraufhin stellt das Webinterface die Sensordaten in einer geordneten Tabelle dar.\par
		
		Der Nutzer selektiert dann den Parameter 'Name', woraufhin das Webinterface die Sensoren nach Namen alphabetisch sortiert.
	\item \textbf{Benutzung mehrerer Webinterface-Instanzen}\\
		Das Webinterface ist nicht geöffnet, es ist ein Kafka Stream namens 'TestStream' hinzugefügt worden und im Browser sind Daten des Webinterfaces gespeichert. Die Ansicht des Webinterfaces ist auf Karlsruhe in Deutschland fokussiert.\\
		Der Nutzer öffnet das Webinterface für jede mögliche Ansicht einmal mehr in unterschiedlichen Tabs oder Fenstern des Browsers.\\
		Das Webinterface lädt auf allen Instanzen zu Beginn die gleichen Ansichten.\par
		
		Anschließend wählt der Nutzer pro Instanz verschiedene Ansichten.\\
		Das Webinterface stellt pro Instanz eine Ansicht dar. Die Ansichten der Instanzen werden nach der Auswahl gleichzeitig angezeigt.
	\item \textbf{Auswahl eines Sensortyps im Webinterface}\\
		Das Webinterface ist geöffnet und es ist ein Kafka Stream namens 'TestStream' hinzugefügt worden. Die Ansicht des Webinterfaces ist auf Karlsruhe in Deutschland fokussiert.\\
		Der Nutzer wählt im Webinterface die darzustellenden \glsdisp{sensortyp}{Sensortypen} aus. Er aktiviert nur den Sensortyp 'Feinstaub'.\\
		Das Webinterface stellt daraufhin nur noch die Werte für 'Feinstaub'-Sensoren dar.\par
	\item \textbf{Export einer Archiv-Datei aus dem Webinterface}\\
		Das Webinterface ist geöffnet und es ist ein Kafka Stream namens 'TestStream' hinzugefügt worden.\\
		Der Nutzer wählt die Option 'Export Sensordata' und wählt das Format 'NetCDF', woraufhin das Webinterface den browserspezifischen Download-Dialog öffnet.\\
		Der Nutzer wählt das Verzeichnis 'SensorExport' bzw. erstellt es, falls es noch nicht existiert, legt den Namen auf 'TestSensorExport' fest und bestätigt.\par
		
		Der Nutzer wählt das Format
		\begin{itemize}
		\item NetCDF
		\end{itemize}
		, wählt das Verzeichnis 'SensorExport' bzw. erstellt es, falls es noch nicht existiert, legt den Namen auf 'TestSensorExport' fest und bestätigt.\\
		Die Konfigurations-GUI exportiert den Stream.
	\item \textbf{Wiederholungen im Webinterface anzeigen}\\
		Das Webinterface ist geöffnet und es ist ein Kafka Stream namens 'TestStream' hinzugefügt worden.\\
		Der Nutzer wählt im Webinterface eine Zeit aus oder verschiebt den Zeitregler 1 Minute zurück. \\
		Das Webinterface stellt daraufhin eine Wiederholung der Sensordaten von der Ausgewählten Zeit bis zum aktuellen Zeitpunkt dar.\par
		
		Der Nutzer kann mit einem Klick auf den 'Pause' Knopf die Wiederholung anhalten. Drückt er anschließend wieder auf 'Play', läuft die Wiederholung weiter.
	\item \textbf{Echtzeitdarstellung im Webinterface}\\
		Das Webinterface ist geöffnet, es ist ein Kafka Stream namens 'TestStream' hinzugefügt worden und das Webinterface zeigt eine Wiederholung an.\\
		Der Nutzer wählt 'Live' aus, woraufhin das Webinterface wieder die Echtzeitdaten darstellt und den Zeitregler wieder zurück auf 'Live' stellt.\par
	\item \textbf{Detailansicht eines Sensors im Webinterface anzeigen}\\
		Das Webinterface ist geöffnet und es ist ein Kafka Stream namens 'TestStream' hinzugefügt worden.\\
		Der Nutzer wählt in der Ansicht einen Sensor aus.\\
		Das Webinterface stellt daraufhin eine detaillierte Auflistung der Sensoreigenschaften dar.\par
	\item \textbf{Melden eines Sensors im Webinterface}\\
		Das Webinterface ist geöffnet und es ist ein Kafka Stream namens 'TestStream' hinzugefügt worden.\\
		Der Nutzer wählt in der Ansicht einen Sensor aus und klickt auf die Schaltfläche 'Report Sensor'. Anschließend schreibt der Nutzer in das Feld 'Reason' den Grund des Meldens an. Hier ist dieser Grund 'TestReportReason'.\\
		Das Webinterface zeigt dem Nutzer daraufhin an, dass die Meldung des Sensors erfolgt ist und speichert sie ab.\par
		
\end{enumerate}