\chapter{Zielbestimmung}
Das Produkt dient der Verarbeitung und Darstellung von Sensordatenstreams. Durch die übersichtliche Visualisierung der Daten auf einer Karte wird die schnelle Analyse von großen Datenmengen ermöglicht und der Zeitaufwand wird minimiert.\\
Ein Hauptmerkmal unseres Produktes ist die Fähigkeit, zusätzlich zu Echtzeitdaten auch historische Datenbestände zu verarbeiten und zu exportieren.
\section{Musskriterien}
\subsection{Backend (Server)}
\begin{enumerate}[label=\textbf{MK\arabic{enumi}0}]
	\setcounter{enumi}{99}
	\item Der Server kann Sensordaten empfangen
	\item Eingeführte Sensordaten werden gesichert
	\item Neue Sensordaten werden zeitnah an alle Instanzen des Webinterfaces weitergeleitet und dargestellt
	\item Der Dienst ist logisch modular aufgebaut und erlaubt das Ergänzen und Ersetzen von einzelnen funktionalen Modulen, wie z.B. verschiedene Exportformate, Zwischenmodule
	\item Der Server verarbeitet und speichert Daten für spätere Verwendung
	\item Der Server kann vorverarbeitete und gespeicherte Daten abrufen
	\item Der Server unterstützt skalar- und vektorwertige Sensortypen
	\item Der Server kann durch eine Admin-GUI gesteuert werden
	\item Der Server unterstützt das Filtern von ausgegebenen und angezeigten Daten
\end{enumerate}
\subsection{Frontend (Webinterface)}
\begin{enumerate}[label=\textbf{MK\arabic{enumi}0}]
	\setcounter{enumi}{199}
	\item Das Webinterface unterstützt die rasterisierte Darstellung der Sensordaten auf einer Weltkarte in Form von vordefinierten Shapes
	\item Das Webinterface unterstützt die Darstellung einer auf Deutschland beschränkten Ansicht
	\item Der Nutzer kann die Karte in vordefinierten Detaillierungsgraden darstellen lassen
	\item Der Nutzer kann aktuelle und historische Sensordaten über das Webinterface darstellen lassen
	\item Der Nutzer kann die Sensordaten über das Webinterface herunterladen
	\item Der Nutzer kann kürzlich beobachtete Daten als Wiederholung anzeigen lassen
	\item Das Webinterface unterstützt die Darstellung von erweiterten Informationen bzgl. der Sensordaten in Form von Graphen
	\item Die Standardsprache des Webinterfaces ist Englisch
	\item Das Webinterface kann parallel von mehreren Nutzern aufgerufen und benutzt werden
\end{enumerate}

\section{Wunschkriterien}
\subsection{Backend (Server)}
\begin{enumerate}[label=\textbf{WK\arabic{enumi}0}]
	\setcounter{enumi}{99}
	\item Der Server skaliert mit unterschiedlich großen Datenmengen
	\item Der Server läuft auch mit fehlerhaften Daten stabil
	\item Der Server überarbeitet im Leerlauf fehlerhafte Daten aus der Datenbank
	\item Der Server unterstützt das Hinzufügen von neuen Anzeigesprachen für das Webinterface
	\item \label{NetCDF_Import}Der Server unterstützt den Import von historischen Daten im NetCDF-Format und kann diese Daten problemlos verarbeiten
	%\item Der Server gibt aussagekräftige Fehlermeldungen aus
\end{enumerate}
\subsection{Frontend (Webinterface)}
\begin{enumerate}[label=\textbf{WK\arabic{enumi}0}]
	\setcounter{enumi}{199}
	% Zoom / Darstellung
	\item Die Genauigkeit der Darstellung von Clustern wird entsprechend der Zoom-Stufe angepasst % MARKIERT
	\item Die Approximation von Clustern führt nicht zu größeren Diskrepanzen oder Wartezeiten
	\item Der Nutzer kann verschiedene Sensordatentypen an- bzw. ausschalten
	% Favoriten
	\item Der Nutzer kann seine Ansicht als Favoriten abspeichern
	\item Der Nutzer kann favorisierte Ansichten wiederherstellen
	% \item Der Nutzer kann einzelne Sensoren/Cluster aus der Darstellung ausschließen
	% Fehler bei Sensoren
	\item Der Nutzer kann Sensoren melden
	\item Dem Nutzer wird darauf hingewiesen, wenn das Abrufen von Sensordaten nicht möglich ist
	\item Enthält ein Cluster keine Sensoren, wird der Cluster nicht angezeigt
	% Aktualisierung
	\item Der Nutzer kann zwischen einer automatischen und einer manuellen Aktualisierung von Echtzeitsensordaten wechseln
	% Benachrichtigungen
	%\item Der Nutzer kann Benachrichtigungen mit Bedingungen einstellen, die dem Nutzer die aktuellen Daten melden, falls die Bedingungen erfüllt sind
	%\item Auf Graphen werden Bedingungen für Benachrichtigungen als Grenzwerte angezeigt
	%\item Der Nutzer kann Töne und Farben für Benachrichtigungen festlegen
	% Sonstiges
	%\item Für die häufigsten Displayauflösungen wird immer eine benutzerfreundliche Darstellung angeboten % https://en.wikipedia.org/wiki/Display_resolution#Common_display_resolutions
	\item Der Nutzer kann eine Anzeigesprache auswählen
	\item Die Anwendung speichert automatisch die Einstellungen und Favoriten des Nutzers über Browsersessions hinweg
\end{enumerate}

\section{Abgrenzungskriterien}
\begin{enumerate}[label=\textbf{AK\arabic{enumi}0}]
	\setcounter{enumi}{99}
	\item Der Server speichert Sensordaten nicht auf unbegrenzte Zeit und in unbegrenzter Menge
	\item Der Server speichert keine Daten von Nutzern und deren Aktivitäten/Interaktionen mit dem Webinterface
	\item Der Datendurchsatz des Servers wird durch lokale Netzwerkgeschwindigkeiten beschränkt
	\item Der Server ist nicht in der Lage, korrekte Vorhersagen zu erstellen
	\item Der Server ist nicht in der Lage, Sensordaten von fehlerhaften Sensoren auszuwerten
	\item Der Server ist nicht in der Lage, unbegrenzt viele Daten anzuzeigen und zu aktualisieren
	\item Der Server ist nicht in der Lage, durch Störungseinflüsse veränderte Sensordaten zu erkennen
	\item Der Server unterstützt keine weiteren Eingabeformate als Apache Kafka Streams (und, falls \ref{NetCDF_Import} erfüllt ist, NetCDF-Dateien)
	\item Der Server unterstützt nicht das manuelle Hinzufügen von neuen Sensordaten
\end{enumerate}