\chapter{Anhang}
	
	\printglossaries
	\glsaddall
    \newglossaryentry{apache_kafka}{name=Apache Kafka, description={Ein Open-Source-Project der Apache Software Foundation. Dient der Speicherung und Verarbeitung von großen Datenströmen}}
    
    \newglossaryentry{client}{name=Client, description={Nimmt angebotene Dienste des Servers wahr}}
    
    \newglossaryentry{cluster}{name=Cluster, description={Eine beliebig große räumliche Ballung und Zusammenfassung von Sensoren und den dazugehörigen Messwerten}}
    
    \newglossaryentry{cluster_shape}{name=Clustershape, description={Eine geometrische Form, die einen bestimmten Raum und darin enthaltene Sensoren überdeckt und durch seine Attribute, beispielsweise Farbe, Sensordaten darstellt}}
    
    \newglossaryentry{cookie}{name=Cookie, description={Eine Textinformation, die durch eine Webseite beim Besucher abgelegt wird. Diese kann im weiteren Verlauf von der Seite verarbeitet werden, beispielsweise zur Nutzeridentifikation oder für die Warenkorbfunktion}}
    
    \newglossaryentry{csv}{name=CSV, description={Comma-separated values (.csv) ist ein Dateiformat und dient der Speicherung und dem Austausch einfach strukturierter Daten}}
    
    \newglossaryentry{datenstrom}{name=Datenstrom, description={Ein kontinuierlicher Fluss von Datensätzen}}
    
    \newglossaryentry{filter}{name=Filter, description={Eine Beschränkung des Datenbestandes nach bestimmten Kriterien}}
    
    \newglossaryentry{frost_server}{name=FROST Server, description={Der Frauenhofer Open Source SensorThings API Server ist eine Implementierung des OGC Standards und ermöglicht die effiziente Verarbeitung von Sensordaten im "Internet of Things"}}
    
    \newglossaryentry{iot}{name=Internet of Things, description={Sammelbegriff für die Vernetzung von Gegenständen mit dem und durch das Internet}}
    
    % \newglossaryentry{json}{name=JSON, description={Die JavaScript Object Notation (.json) ist ein kompaktes Datenformat zum Datenauschtausch zwischen Anwendungen}}
    
    \newglossaryentry{mqtt}{name=MQTT, description={Message Queue Telemetry Transport ist ein offenes Nachrichtenprotokoll für Maschine-zu-Maschine Kommunikation}}
    
    \newglossaryentry{netcdf_glo}{name=NetCDF, description={Das Network Common Data Format (.nc, .cdf) ist ein standardisiertes Dateiformat für den Austausch wissenschaftlicher Daten}}
    
    \newglossaryentry{ogc}{name=OGC, description={Das Open Geospatial Consortium, eine gemeinnützige Organisation die sich auf Standardisierungen bezüglich der computergestützten Verarbeitung insbesondere von Geodaten spezialisiert}}
    
    \newglossaryentry{outlier}{name=Outlier, description={Ein Messwert oder ein Befund, der nicht in eine erwartete Messreihe passt}}
    
    \newglossaryentry{raster}{name=Raster, description={Eine Aufteilung der Karte in Shapes zur besseren Visualisierung}}
    
    % \newglossaryentry{responsive_webdesign}{name=Responsive Webdesign, description={Ein gestalterisches und technisches Paradigma, dass die Erstellung von Webseiten auf Flexibilität auslegt und auf die Merkmale der Endgeräte reagieren lässt}}
   
    \newglossaryentry{sensor}{name=Sensor, description={Ein technisches Bauteil, dass physikalische oder chemische Eigenschaften seiner Umgebung qualitativ oder quantitativ erfassen kann}}
    
    % Oder lieber Datentyp ?
    \newglossaryentry{sensordatentyp}{name=Sensordatentyp, description={Die Art des Messwerts, der durch einen Sensor erfasst wird, z.B Feinstaub, Temperatur, Luftfeuchtigkeit oder Windstärke}}
    
    \newglossaryentry{server}{name=Server, description={Ein Computer/Computerprogramm, dass Dienstprogramme, Daten oder andere Ressourcen für Clients bereitstellt}}
    
    % \newglossaryentry{split_panel}{name=Split Panel, description={Die Darstellung zweier oder mehr Datensätze nebeneinander zum besseren Vergleich}}
    
    \newglossaryentry{webinterface}{name=Webinterface, description={Eine Programmschnittstelle, die über einen Internetbrowser angesprochen werden kann}}