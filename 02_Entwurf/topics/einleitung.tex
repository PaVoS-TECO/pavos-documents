\chapter{Einleitung}

Dieses Dokument ist das Ergebnis der Entwurfsphase und soll einen Überblick über den Entwurf aller Teilelemente des PaVoS-Projektes geben. Diese sind im Rahmen der im Pflichtenheft definierten Anforderungen entstanden. Dabei urde eine Packethierarchie und dazugehörende Klassen und Schnittstellen erzeugt, die wiederum einen Rahmen für die kommende Implementierungsphase bilden. Das gesamtprojekt wurde dazu in verschiedene einzelne Elemente aufgeteilt.\\\\
Die zentralen Elemente sind dabei:
\begin{enumerate}
	\item \textbf{Die Bridge} vom Frost-Server zu Kafka.
	\item \textbf{Der Kern}, der direkt mit Kafka und den Streams arbeitet.
	\item \textbf{Die Webansicht} für den Nutzer im Browser.
\end{enumerate}
Dazu gibt es noch verschiedene Elemente die zwischen diesen Erstgenannten arbeiten, oder den Datenaustausch dieser übernehmen. Das sind folgende Elemente:
\begin{enumerate}
	\item \textbf{Der Import} dient dazu Datenbestände in PaVoS einzuschleusen.
	\item \textbf{Die DatabaseConnection} dient dazu Daten aus Kafka für die Karte der Webansicht bereitzustellen.
	\item \textbf{Der DataTransferControl} dient dazu Daten aus Kafka für die Grafiken der Webansicht bereitzustellen.
	\item \textbf{Der Export} dient dazu Daten aus Kafka in Dateien zu schreiben und diese der Webansicht zuzusenden.
\end{enumerate}
Jedes dieser Elemente stellt ein oder mehrere Packages dar.\\\\
Als Entwurfsumgebung wurde StarUML verwendet. Die Entwicklung soll für alle Serverseitigen Elemente in Java erfolgen während bei der Webansicht auch Javascript zum Einsatz kommen wird.\\\\
Der Entwurf besteht aus insgesamt 118 Klassen und 10 Schnittstellen. Diese werden alle in diesem Dokument im Detail behandelt aber auch in mithilfe der Klassendiagramme und einzelner Sequenzdiagramme in ihrem Kontext dargestellt und deren Funktionsweise erläutert.
